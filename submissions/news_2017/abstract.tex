%\documentstyle[12pt]{article}
\documentclass[12pt]{article}

\begin{document}

\begin{center}
{\large\bf Towards a better predictive model from rest fMRI: benchmarks across
multiple phenotypes}\\[4mm]
\end{center}

\begin{center}
Kamalaker Dadi$^a$$^b$, Darya Chyzhyk$^a$$^b$, Alexandre Abraham$^a$$^b$,
Mehdi Rahim$^a$$^b$, Bertrand Thirion$^a$$^b$, Ga\"el Varoquaux$^a$$^b$\\
\end{center}

\begin{center}
{\small \em $^a$Parietal Team, INRIA, Paris-Saclay
University, France.}\\
{\small \em $^b$CEA, DSV, I$^{2}$BM, Neurospin Center,
Paris-Saclay University, 91191, Gif-sur-Yvette, France.}\\
\end{center}

% Here is the abstract.
\begin{abstract}
	Psychiatry and psychology are based on assessing individuals traits,
    characterized through behavioral testing and questionnaires. Imaging of
    brain activity raises the hope of measuring the physiological differences
    that underlie these psychological variations \cite{smith} \cite{miller}. In
    \cite{abraham}, we have
    introduced an automated pipeline capable of learning this link across
    individuals using large cohorts of functional magnetic resonance images
    acquired during rest (Rest fMRI). We present an openly available
    implementation of this pipeline and how we used it to draw best practices
    from its application on various problems. Rest fMRI is a promising
    universal marker of brain function \cite{bharat}, as it can easily be acquired on
    many different individuals and is applicable to disease populations. It is
    used to capture functional-connectivity information, i.e. interaction
    patterns in brain activity. The challenge is then to relate it to behavior
    and pathology. Our pipeline successively defines regions from rest fMRI,
    build connectomes from time series signals extracted upon on these regions
    of interests, and compares connectomes across subjects using machine
    learning. We applied it on five datasets to i) determine the steps to
    obtain the best prediction, and ii) predict phenotypic information with
    good accuracy. Through systematic comparisons, we outline dominant choices
    for each pipeline step. Our results show that this analysis pipeline can
    be adapted to various psychological questions for instance in
    epidemiological studies \cite{miller}, moving imaging closer to a diagnosis tools in
    clinical settings.

     

%    [1] Smith, Stephen et al. A positive-negative mode of population
%    covariation links brain connectivity, demographics and behavior. Nature
%    Neuroscience, 2015.

%    [2] Miller Karla et al. Multimodal population brain imaging in the UK
%    Biobank prospective epidemiological study. Nature Neuroscience, 2016.

%    [3] Bharat Biswal et al. Toward discovery science of human brain function.
%    PNAS, 2010.
%    [4] Alexandre Abraham et al. Deriving reproducible biomarkers from
%    multi-site resting-state data: An Autism-based example. NeuroImage, 2016.

\end{abstract}
%----
\begin{thebibliography}{9}
    \bibitem{smith} 
        Smith, Stephen et al. 
        \textit{A positive-negative mode of population covariation links brain
        connectivity, demographics and behavior}. 
        Nature Neuroscience, 2015.
         
    \bibitem{miller} 
        Miller Karla et al. 
        \textit{{Multimodal population brain imaging in the UK Biobank prospective
    epidemiological study}}. Nature Neuroscience, 2016.
         
    \bibitem{bharat} 
        Bharat Biswal et al.
        \textit{Toward discovery science of human brain function}.
        Proceedings of the National Academy of Sciences, 2010.

    \bibitem{abraham} 
        Alexandre Abraham et al.
        \textit{Deriving reproducible biomarkers from multi-site
        resting-state data: An Autism-based example}.
        Neuroimage, 2016.
\end{thebibliography}
\end{document}
